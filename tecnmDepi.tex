% Plantilla diseñada por
% Gerardo Marx Chávez-Campos para el TeCNM
\documentclass{beamer}
\usetheme{tecnm}
\usepackage{url}
\usepackage[spanish]{babel}
\usepackage[utf8]{inputenc}
\usepackage[T1]{fontenc}
\usepackage{graphicx}
\usepackage{mathrsfs}
\usepackage{amsmath}
\usepackage{multimedia}
\usepackage{rotating}
\newtheorem{axioma}{Teorema}

\newcommand{\fst}[2]{\subsection{#2}\frame{\frametitle{#2} #2}}
%Comando para crear un frame con imagen:
\newcommand\imageFrame[2]{
\begingroup
\begin{frame}
  \begin{center}
\includegraphics[width=3.5in]{Figures/#1}\\
\Large #2
    \end{center}
\end{frame}
\endgroup
}


\title[Título corto]{Título largo }
\date{Abril -- 2018}
%\date{\today}
\author[TecNM]{Gerardo Marx Chávez-Campos}
\institute[TecNM]{Tecnológico Nacional de México\\
Instituto Tecnológico de Morelia\\
División de Estudios de Posgrado e Investigación}
\logo{\includegraphics[width=.4in]{Figures/soloLogoTecnm}}
\begin{document}

%----INICIO Portada
\begin{frame}[plain]
  \titlepage
  \vspace{-.6in}
  \includegraphics[width =0.25 \textwidth ]{Figures/soloLogoTecnm}
\end{frame}
%---FIN Portada

%---INICIO Contenido%
\begin{frame}[fragile]
\frametitle{Contenido}
  \tableofcontents
\end{frame}
%---FIN Contenido%

%-----------------------
\section{Objetivo general}
\begin{frame}
	\centering
	\textbf{Objetivo general} de los puntos  que se revisan en la reunión, \alert{es lograr los indicadores que CONACyT evalúa} para permanecer o ingresar en el \textbf{PNPC}.\\

	
\end{frame}
%-------------




\end{document}
